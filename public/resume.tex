\documentclass[5pt,a4paper]{article}
\usepackage[english]{babel}
\usepackage[a4paper,top=2.5cm,left=2.5cm,right=2.5cm,bottom=2.5cm]{geometry}
\usepackage{currvita}
\usepackage[hidelinks]{hyperref}
\usepackage{multicol} 
\usepackage{enumitem}
\usepackage{comment}
\usepackage[super]{nth}
\usepackage[left]{eurosym}
\setlist[itemize]{noitemsep}
\setlength{\columnseprule}{1pt}
\newcommand*{\ac}[1]{\mbox{#1}}
\tolerance=600
\urlstyle{same}
\pagestyle{empty}
\date{}
\begin{document}
\begin{cv}{}
\setlength\cvlabelwidth{50pt}

  \begin{multicols}{2}
	[
	\begin{center}
	\section*{Jan Mazza}
	\end{center}
	]
\begin{flushright}
	\normalsize Via delle Fontanelle, 18\\
    50014 Fiesole, Florence\\
		Italy\\
  (+39) 3394642168\\
\end{flushright}
  
\begin{flushleft}
  \href{https://scholar.google.com/citations?hl=en&user=sbrAYfUAAAAJ}{Google Scholar}\\ 
  \href{https://www.linkedin.com/in/jan-mazza/}{LinkedIn}\\
     jan.mazza@eui.eu\\
		\url{www.janmazza.com} \\
\end{flushleft}
		\end{multicols}
		\vspace{1pt}
		
\begin{cvlist}{Education}
	\hrule
	\item[\small 2019 -- ] \normalsize PhD in Economics\\
	\footnotesize EUROPEAN UNIVERSITY INSTITUTE
	\normalsize
	\begin{itemize}
	\item Supervisors: Ramon Marimon (\nth{1}), Thomas Crossley (\nth{2})
	\item Research interests: inter-generational inequality, household finance, international macro, macro-finance, health
 \end{itemize}
 	\item[\small 2022 -- ] \normalsize Visiting Researcher (European Doctoral Program in Quantitative Economics)\\
	\footnotesize POMPEU FABRA UNIVERSITY
	\item[\small 2019 -- 2020] \normalsize MRes in Economics\\
	\footnotesize EUROPEAN UNIVERSITY INSTITUTE
	\item[\small 2016 -- 2017] \normalsize MSc in Economics and Philosophy (Merit)\\
	\footnotesize LONDON SCHOOL OF ECONOMICS AND POLITICAL SCIENCE
  \normalsize \item[\small 2014 -- 2016] \normalsize Master's Degree in Economics (Summa cum laude)\\
	\footnotesize UNIVERSITY OF BOLOGNA
    \normalsize \item[\small 2016] \normalsize Visiting Student \\ 
    \footnotesize  LUDWIG-MAXIMILI\"ANS UNIVERSITY OF MUNICH
	\normalsize \item[\small 2011 -- 2014] \normalsize Bachelor's Degree in Economics, Markets and Institutions (Summa cum laude)\\
	\footnotesize UNIVERSITY OF BOLOGNA
    \normalsize \item[\small 2012 -- 2013] \normalsize Visiting Student\\ 
    \footnotesize PANTH\'EON-ASSAS PARIS-2 UNIVERSITY 
  \end{cvlist}

  
%%%%%%%%%%%%%%%%%%%%%%%%%%%%%%%%%%%	
%%%% RESEARCH EXPERIENCE %%%%%%
%%%%%%%%%%%%%%%%%%%%%%%%%%%%%%%%%%%

  \begin{cvlist}{Research Experience}
  \item[\small 2023 -- ] \normalsize Research Fellow \\
  \footnotesize UNIVERSITY OF BOLOGNA
  \item[\small 2023] \normalsize Visiting Research Officer\\
  \footnotesize INTERNATIONAL MONETARY FUND
	\item[\small 2021 -- 2022] \normalsize Academic Collaborator to prof. David Levine\\
	\footnotesize ROBERT SCHUMAN CENTRE FOR ADVANCED STUDIES (EUI) \small  	
	\item[\small 2021] \normalsize Research Assistant to prof. Thomas Crossley\\
	\footnotesize EUROPEAN UNIVERSITY INSTITUTE \small  
	\item[\small 2018 -- 2019] \normalsize Research Assistant\\
	\footnotesize BRUEGEL \small
 \end{cvlist}
  

%%%%%%%%%%%%%%%%
%%% PAPERS %%%%%
%%%%%%%%%%%%%%%%
\hrule
\begin{cvlist}{Publications}
\item[\small 2023] 
\href{https://bmcmedresmethodol.biomedcentral.com/articles/10.1186/s12874-023-01948-y}{\textbf{Does the Feedback of Blood Results in Observational Studies Influence Response and Consent? A Randomised Study of the Understanding Society Innovation Panel}} (joint with Michaela Benzeval, Alexandria Andrayas, Tarek Al Baghal, Jonathon Burton, Thomas F.\ Crossley and Meena Kumari)\\ \textit{BMC Medical Research Methodology, (2023) 23:134}\\
\begin{comment}    
    \footnotesize \textbf{Background}: While medical studies generally provide health feedback to participants, in observational studies this is not always the case due to logistical and financial difficulties, or concerns about changing observed behaviours. However, evidence suggests that lack of feedback may deter participants from providing biological samples. This paper investigates the effect of offering feedback of blood results on participation in biomeasure sample collection.
    \textbf{Methods}: Participants aged 16 and over from a longitudinal study – the Understanding Society Innovation Panel - were randomised to three arms – nurse interviewer, interviewer, web survey – and invited to participate in biomeasures data collection. Within each arm they were randomised to receive feedback of their blood results or not. For those interviewed by a nurse both venous and dried blood samples (DBS) were taken in the interview. For the other two arms, they were asked if they would be willing to take a sample, and if they agreed a DBS kit was left or sent to them so the participant could take their own sample and return it. Blood samples were analysed and, if in the feedback arms, participants were sent their total cholesterol and HbA1c results. Response rates for feedback and non-feedback groups were compared: overall; in each arm of the study; by socio-demographic and health characteristics; and by previous study participation. Logistic regression models of providing a blood sample by feedback group and data collection approach controlling for confounders were calculated.
    \textbf{Results}: Overall 2162 (80.3\% of individuals in responding households) took part in the survey; of those 1053 (48.7\%) consented to provide a blood sample. Being offered feedback had little effect on overall participation but did increase consent to provide a blood sample (unadjusted OR 1.38; CI: 1.16–1.64). Controlling for participant characteristics, the effect of feedback was highest among web participants (1.55; 1.11–2.17), followed by interview participants (1.35; 0.99 –1.84) and then nurse interview participants (1.30; 0.89–1.92).
    
    \textbf{Conclusions}: Offering feedback of blood results increased willingness to give samples, especially for those taking part in a web survey.
\end{comment}
\end{cvlist}
%%%%%%% WORKING PAPERS %%%%%%%%%%%%%%%%%%%%%%
\begin{cvlist}{Working Papers}
\item[] \normalsize
    \href{http://www.janmazza.com/Mazza_Inheritance_Expectations_Education.pdf}{\textbf{Inheritance Expectations, Dynastic Altruism, and Education}}\\
    \footnotesize \input{../../Inheritance_expectation/paper/abstract.tex}
    % \footnotesize Economic and demographic factors underpin the rising importance of inheritance flows across advanced economies. Based on Italian data about expected and realized inter-generational asset transfers, this paper shows that their influence extends before transfers and beyond assets, uncovering a strong, positive conditional association between inheritance expectations and the pursuit of higher education. This sssociation is driven by expected housing wealth, pointing at long-run determinants. The intention to leave a bequest, in fact, is strongly associated with having received one, or the expectation to receive one in the future, consistently with heterogeneity and persistence in dynastic altruism. I rationalize the empirical findings through a simple analytical model where dynastic altruism, connecting anticipated inheritances with bequest motives, shifts the inter-temporal trade-off associated with education, thus perpetuating inter-generational disparities in education and income. Through a richer quantitative lifecycle model, I show that heterogeneity in bequest motives can account for almost 40\% of the observed gap in student rates. Finally, through a set of counterfactual exercises, I illustrate how the strength of the positive link between inheritance expectations and education critically rests on high short-term costs and low expected long-term benefits of education within the lifecycle.
    \item[] \normalsize
    \href{http://www.janmazza.com/Mazza_Mijakovic_Domestic_Inequality_Global_Imbalances.pdf}{\textbf{Domestic Inequality and Global Imbalances}} (joint with Andrej Mijakovic)\\    
    \footnotesize \input{../../MM_ca_ineq/04_drafts/abstract.tex}
\end{cvlist}
%%%%%%% WORK IN PROGRESS %%%%%%%%%%%%%%%%%%%%
\begin{cvlist}{Work in Progress}
    \item[] \normalsize
    \textbf{Classical and Non-Classical Measurement Errors in Blood Pressure Measures from Understanding Society: an Estimation} (joint with Thomas F.\ Crossley)\\
    \item[] \normalsize
    \textbf{Death by Waiting? Treatment Delays, Emergency Department Congestion and Patients’ Outcomes} (joint with Matteo Lippi Bruni, Cristina Ugolini, and Rossella Verzulli)\\
\begin{comment}        
        \footnotesize In surveys, it is seldom possible to obtain the most accurate available measure of any variable of interest, given the high costs (monetary and not) associated to the necessary procedures. In the context of Understanding Society, we consider the setting of blood pressure measurements as gathered by nurses and compare them with self-taken and interviewer-taken ones. Assuming classical measurement error for nurse-taken measures and non-classical error for the other two sources, we estimate the parameters of interest through Generalised Method of Moments (GMM) in order to correct for any possible bias in measures exhibiting non-classical error.
\end{comment}
\end{cvlist}
%%%%%%%% POLICY PUBLICATIONS %%%%%%%%%%%%%%
\begin{cvlist}{Policy Publications}	
\item[\small 2019] \small \normalsize 
    \textbf{Added Value in Cohesion Policy: Learning from the Programme Characteristics that Produce the Best Results} (joint with Zsolt Darvas, Antoine Mathieu-Collin and Catarina Midoes)\\
    \textit{Prepared for the Regional Development Committee (REGI) of the European Parliament, 2019}

\item[\small 2018] \normalsize 
    \textbf{A Monetary Policy Framework for the ECB to Deal with Uncertainty} (joint with Gregory Claeys and Maria Demertzis)\\
    \textit{Bruegel Policy Contribution No 2018/21, prepared for the Economic and Monetary Affairs Committee (ECON) of the European Parliament, 2018}

\end{cvlist}

%%%%%%%%%%%%%%%%%%%%%%%%%%%%%%%%%%%%%%%%%%%%
%%%%% CONFERENCES, SEMINARS, WORKSHOPS %%%%%
%%%%%%%%%%%%%%%%%%%%%%%%%%%%%%%%%%%%%%%%%%%%
\hrule
 \begin{cvlist}{Conferences, Workshops and Seminars (including scheduled)}
    \item[\small 2024] \normalsize UPF-CREI Macroeconomics Lunch, \nth{1} Imperial College PhD Conference in Economics and Finance, \nth{27} Theories and Methods in Macro -- T2M (Amsterdam), PSE-CEPR Policy Forum 2024, EDP Jamboree (Barcelona), UCL Stone Centre PhD Conference on Income and Wealth Inequality, EuHEA Conference 2024 (Vienna), \nth{3} PhD and Post-Doctoral Workshop in Economics and Finance (Naples)
   \item[\small 2023] \normalsize \nth{12} PhD Student Conference
on International Macroeconomics (Paris Nanterre), PhD Workshop in Economics of Education (Helsinki), EALE 2023 (Prague), Annual Meeting of the Central Bank Research Association - CEBRA (New York)
	\item[\small 2022] \normalsize UPF Applied Lunch Seminar, BSE Jamboree
  \end{cvlist}

%%%%%%%%%%%%%%%%%%%%%%%%%%%%%%%%%%%%%%%
%%%%%% TEACHING EXPERIENCE %%%%%%%%%%%%
%%%%%%%%%%%%%%%%%%%%%%%%%%%%%%%%%%%%%%
  \hrule 
  \begin{cvlist}{Teaching Experience}
  \item[\small 2023] \normalsize Maths and Economics Refresher Course -- \small Teaching Associate \\ 
  \footnotesize SCHOOL OF TRANSNATIONAL GOVERNANCE (EUI)
    \item[\small 2023] \normalsize Economics II: Macro and International Economics -- \small Teaching Assistant\\
	    \footnotesize SCHOOL OF TRANSNATIONAL GOVERNANCE (EUI) -- Prof. Georgios Papakonstantinou \small  
        \item[\small 2021 -- 2023] \normalsize Crisis Seminar -- \small Teaching Assistant\\
	    \footnotesize SCHOOL OF TRANSNATIONAL GOVERNANCE (EUI) -- Prof. Georgios Papakonstantinou \small  
    \item[\small 2022] \normalsize Data for Policy Analysis -- \small Teaching Associate\\
    \footnotesize SCHOOL OF TRANSNATIONAL GOVERNANCE (EUI)% \small  
\end{cvlist}



%%%%%%%%%%%%%%%%%%%%%%%%%%%%%%%%%%%	
%%%% PROFESSIONAL EXPERIENCE %%%%%%
%%%%%%%%%%%%%%%%%%%%%%%%%%%%%%%%%%%
\hrule
  \begin{cvlist}{Other Professional Experience}
	\item[\small 2018] \normalsize Intern in EU Public Affairs\\
	\footnotesize ASSONIME (Association of Italian Joint Stock Companies)\small
	\item[\small 2017 -- 2018] \normalsize Blue Book Trainee\\
	\footnotesize EUROPEAN COMMISSION - DG BUDGET\\
	\small Unit B1 (Multi-annual financial framework) and B4 (Revenue policy and Control of other own resources)
	\end{cvlist}
	
	\hrule
  \begin{cvlist}{Grants}
  \item[\small 2019 -- 2023] \normalsize Polish Ministry of National Education PhD grant
	\item[\small 2015] \normalsize \footnotesize ERASMUS+ EXCHANGE PROGRAMME \normalsize scholarship
  \item[\small 2012] \normalsize University of Bologna scholarship awarded for \footnotesize ACADEMIC EXCELLENCE \normalsize in 2012 (among the best students of the University)
	\item[\small 2012] \normalsize \footnotesize ERASMUS EXCHANGE PROGRAMME \normalsize scholarship
  \end{cvlist}
	\hrule
  \begin{cvlist}{Language and IT skills, certificates}
  \item[Languages] \footnotesize ITALIAN (native), ENGLISH, FRENCH, SPANISH (fluent)
	\item[Software] \footnotesize MICROSOFT OFFICE, STATA, R, JULIA, \LaTeX: \normalsize advanced
	\item \footnotesize PYTHON, MATLAB: \normalsize proficient
	\item[GRE] Quantitative reasoning: 170/170
  \end{cvlist}

\end{cv}
\end{document}
\endinput